%%%% CAPÍTULO 3: MATERIAL E MÉTODOS (PODE SER OUTRO TÍTULO, CONFORME O TRABALHO
%%%% REALIZADO)
%%
%% Deve apresentar modelos utilizados, modelagem empregada, simplificações
%% necessárias, metodologia e descrição do método de cálculo utilizado no
%% desenvolvimento da pesquisa, para que a mesma possa ser reconstituída. Devem
%% ser descritos também: montagem experimental, metodologia para a obtenção de
%% resultados, análise de erros, amostras de resultados obtidos e comentários.
%% Atenção: esta parte pode ser dividida em mais seções primárias conforme a
%% especificidade do assunto.

%% Locais (pastas) de ilustrações deste capítulo
\graphicspath{%
  {./Chapter-3/}%% Primário
%   {./Chapter-3/Illustrations/}%% Secundário (descomentar se houver)
}

\chapter{Material e Métodos}%
\label{chpt:matl-meth}

Deve apresentar modelos utilizados, modelagem empregada, simplificações necessárias, metodologia e descrição do método de cálculo utilizado no desenvolvimento da pesquisa, para que a mesma possa ser reconstituída.
Devem ser descritos também: montagem experimental, metodologia para a obtenção de resultados, análise de erros, amostras de resultados obtidos e comentários.
Atenção: esta parte pode ser dividida em mais seções primárias conforme a especificidade do assunto.

%% Notas:
%% 1. O capítulo de exemplo exibe informações e exemplos sobre comandos e
%%    ambientes TeX/LaTeX, assim como os específicos do pacote utfpr-thesis,
%%    entre outros. Estes comandos e ambientes podem ser utilizados para
%%    produzir as diversas estruturas necessárias à elaboração do documento:
%%    enumerações; citações e referências; equações; ilustrações; tabelas;
%%    abreviaturas, siglas e acrônimos; símbolos; glossário; apêndices e anexos;
%%    índice; entre outras. Observar comentários em todos os arquivos-fonte do
%%    projeto para entender configurações e formatações no modelo.
%% 2. O arquivo final (PDF) pode ser convertido para PDF/A usando diversas
%%    ferramentas, por exemplo:
%%      https://www.pdfforge.org/online/en/pdf-to-pdfa
