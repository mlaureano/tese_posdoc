%%%% CAPÍTULO 1: INTRODUÇÃO
%%
%% Deve apresentar uma visão global da pesquisa, incluindo: breve histórico,
%% importância e justificativa de escolha do tema, delimitações do assunto,
%% formulação de hipóteses, objetivos da pesquisa e estrutura do trabalho.

%% Locais (pastas) de ilustrações deste capítulo
\graphicspath{%
  {./Chapter-1/}%% Primário
%   {./Chapter-1/Illustrations/}%% Secundário (descomentar se houver)
}

\chapter{Introdução}%
\label{chpt:intro}

Deve apresentar uma visão global da pesquisa, incluindo: breve histórico, importância e justificativa de escolha do tema, delimitações do assunto, formulação de hipóteses, objetivos da pesquisa e estrutura do trabalho.

%% Notas:
%% 1. O capítulo de exemplo exibe informações e exemplos sobre comandos e
%%    ambientes TeX/LaTeX, assim como os específicos do pacote utfpr-thesis,
%%    entre outros. Estes comandos e ambientes podem ser utilizados para
%%    produzir as diversas estruturas necessárias à elaboração do documento:
%%    enumerações; citações e referências; equações; ilustrações; tabelas;
%%    abreviaturas, siglas e acrônimos; símbolos; glossário; apêndices e anexos;
%%    índice; entre outras. Observar comentários em todos os arquivos-fonte do
%%    projeto para entender configurações e formatações no modelo.
%% 2. O arquivo final (PDF) pode ser convertido para PDF/A usando diversas
%%    ferramentas, por exemplo:
%%      https://www.pdfforge.org/online/en/pdf-to-pdfa
