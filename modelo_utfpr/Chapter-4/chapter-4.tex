%%%% CAPÍTULO 4: RESULTADOS E DISCUSSÃO
%%
%% Deve descrever detalhadamente os dados obtidos no trabalho. Os resultados são
%% normalmente discutidos a partir de ilustrações (gráficos, quadros, etc.),
%% tabelas, entre outros elementos, que podem ser incluídos no documento. Deve
%% efetuar a comparação dos dados obtidos e/ou resultados com aqueles descritos
%% na revisão da literatura, incluindo os comentários sobre os estudos de outros
%% trabalhos.

%% Locais (pastas) de ilustrações deste capítulo
\graphicspath{%
  {./Chapter-4/}%% Primário
%   {./Chapter-4/Illustrations/}%% Secundário (descomentar se houver)
}

\chapter{Resultados e Discussão}%
\label{chpt:rslt-disc}

Deve descrever detalhadamente os dados obtidos no trabalho.
Os resultados são normalmente discutidos a partir de ilustrações (gráficos, quadros, etc.), tabelas, entre outros elementos, que podem ser incluídos no documento.
Deve efetuar a comparação dos dados obtidos e/ou resultados com aqueles descritos na revisão da literatura, incluindo os comentários sobre os estudos de outros trabalhos.

%% Notas:
%% 1. O capítulo de exemplo exibe informações e exemplos sobre comandos e
%%    ambientes TeX/LaTeX, assim como os específicos do pacote utfpr-thesis,
%%    entre outros. Estes comandos e ambientes podem ser utilizados para
%%    produzir as diversas estruturas necessárias à elaboração do documento:
%%    enumerações; citações e referências; equações; ilustrações; tabelas;
%%    abreviaturas, siglas e acrônimos; símbolos; glossário; apêndices e anexos;
%%    índice; entre outras. Observar comentários em todos os arquivos-fonte do
%%    projeto para entender configurações e formatações no modelo.
%% 2. O arquivo final (PDF) pode ser convertido para PDF/A usando diversas
%%    ferramentas, por exemplo:
%%      https://www.pdfforge.org/online/en/pdf-to-pdfa
