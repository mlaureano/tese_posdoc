%%%% ELEMENTOS PRÉ-TEXTUAIS
%%
%% Parte que antecede o texto com informações que ajudam na identificação e na
%% utilização do trabalho.

%% Folha de Rosto
%% Contém os elementos essenciais à identificação do trabalho, assim como uma
%% Licença Creative Commons (CC) (https://creativecommons.org/choose/):
%% - {TitlePage*} aplica caixa alta no título em idioma secundário.
\begin{TitlePage}%% Argumentos (4):
[BY]%% Tipo de licença (BY, BY-SA, BY-ND, BY-NC, BY-NC-SA ou BY-NC-ND)
% [\YearNum]%% Ano de criação (formato CC)
% [P. Sobrenome-A1]%% Criador(es) do documento (formato CC)
% [Texto da licença]%% Substitui o texto padrão de cada tipo (formato UTFPR)
%%%% Descrição do trabalho (padrão; alterar se necessário)
\DocumentTypeName\ apresentad\ifbool{Graduate}{a}{o} como requisito para obtenção do título de \StudentsTitlesList\ em \CourseName\ da \ifbool{MakeAcr}{\intldescr{UTFPR} (\intl{UTFPR})}{\UTFPRName\ (UTFPR)}\ifbool{DScOrMSc}{. Área de Concentração: Nome da Área}{}.
%%%% Orientador(es) (de 1 a 3): {Número}; {Dados}
% \Advisor{1}{%
%   Gender   = {Male},%% Ou {Female}
%   Title    = {\ProfCall\ \PhDCall},%% {\<Prof/PhD/DSc/MSc/Eng>Call}
%   Fullname = {Prenome{(s)} Sobrenome-B1},%% Conforme o Currículo Lattes
%   Email    = {advisor1@domain},%% Opcional
%   Lattes   = {0000000000020001},%% Opcional
%   ORCID    = {0000-0000-0002-0001},%% Opcional (CHKTEX 8)
% }
%%%% Ficha catalográfica (somente para Teses e Dissertações em catálogo físico):
%%%% - [Local do PDF] (pasta padrão ./Pre-Textual/Extras/);
%%%% - {Nome do PDF} (modelo em ./Pre-Textual/Extras/).
% \IndexCardPDF{doc-index-card.pdf}
\end{TitlePage}

%% Errata (elemento opcional; editar o {Arquivo} para alterar)
% %%%% ERRATA (ELEMENTO OPCIONAL)
%%
%% Lista dos erros ocorridos no texto, seguidos das devidas correções:
%% - {Errata*} insere a autorreferência do documento.
\begin{Errata}%[Título Alternativo]%% Substitui o título padrão
%%%% Formato (com \midrule entre linhas): Página(s) & Onde se lê & Leia-se \\
\labelcpageref{err:chpt-1,err:chpt-2,err:chpt-3,err:chpt-4,err:chpt-5,err:chpt-6} &
capítulo{(s)}                                                                     &
seção{(ões)} primária{(s)}                                                        \\ \midrule%
\pageref{err:ssect}         &
subseção{(ões)}             &
seção{(ões)} terciária{(s)} \\
\end{Errata}


%% Folha de Aprovação
%% Contém os elementos essenciais à aprovação do trabalho (sem as assinaturas).
%%%% Opção 1 (gerada por meio do pacote utfpr-thesis):
%%%% - {ApprovalPage*} insere a titulação após o nome do membro da banca.
\begin{ApprovalPage}%% Argumentos (4):
% [brazilian]%% Idioma original ou primário (brazilian ou english)
% [DD de mmmmmm de YYYY]%% Data de defesa (dia, mês por extenso e ano)
% [DD/MM/YYYY]%% Data de defesa (forma abreviada para mestrado e doutorado)
% [\textwidth]%% Largura de linha de assinatura (graduação e especialização)
%%%%%% Descrição do trabalho (padrão; alterar se necessário)
\DocumentTypeName\ apresentad\ifbool{Graduate}{a}{o} como requisito para obtenção do título de \StudentsTitlesList\ em \CourseName\ da \ifbool{MakeAcr}{\intldescr{UTFPR} (\intl{UTFPR})}{\UTFPRName\ (UTFPR)}\ifbool{DScOrMSc}{. Área de Concentração: Nome da Área}{}.
%%%%%% Membro(s) da Banca Examinadora (de 3 a 6): {Número}; {Dados}
% \Member{1}{%
%   Gender      = {Male},%% Ou {Female}
%   Title       = {\ProfCall\ \PhDCall},%% {\<Prof/PhD/DSc/MSc/Eng>Call}
%   Fullname    = {Prenome{(s)} Sobrenome-C1},%% Conforme o Currículo Lattes
%   Lattes      = {0000000000030001},%% Opcional
%   ORCID       = {0000-0000-0003-0001},%% Opcional (CHKTEX 8)
%   Institution = {Instituição (Membro-C1)},%% Nome completo e por extenso
% }
% \Member{2}{%
%   Gender      = {Male},%% Ou {Female}
%   Title       = {\ProfCall\ \PhDCall},%% {\<Prof/PhD/DSc/MSc/Eng>Call}
%   Fullname    = {Prenome{(s)} Sobrenome-C2},%% Conforme o Currículo Lattes
%   Lattes      = {0000000000030002},%% Opcional
%   ORCID       = {0000-0000-0003-0002},%% Opcional (CHKTEX 8)
%   Institution = {Instituição (Membro-C2)},%% Nome completo e por extenso
% }
% \Member{3}{%
%   Gender      = {Male},%% Ou {Female}
%   Title       = {\ProfCall\ \PhDCall},%% {\<Prof/PhD/DSc/MSc/Eng>Call}
%   Fullname    = {Prenome{(s)} Sobrenome-C3},%% Conforme o Currículo Lattes
%   Lattes      = {0000000000030003},%% Opcional
%   ORCID       = {0000-0000-0003-0003},%% Opcional (CHKTEX 8)
%   Institution = {Instituição  (Membro-C3)},%% Nome completo e por extenso
% }
\end{ApprovalPage}
%%%% Opção 2 (gerada a partir do Sistema Acadêmico ou da secretaria):
%%%% - [Local do PDF] (pasta padrão ./Pre-Textual/Extras/);
%%%% - {Nome do PDF} (modelos em ./Pre-Textual/Extras/).
% \ApprovalPagePDF{doc-approval-page.pdf}

%% Dedicatória (elemento opcional; editar o {Arquivo} para alterar)
% %%%% DEDICATÓRIA (ELEMENTO OPCIONAL)
%%
%% Texto (pessoal) em que se presta homenagem ou se dedica o trabalho.
\begin{Dedication}%% Argumentos (2):
% [0.5\textheight]%% Deslocamento vertical a partir da margem superior
% [Título]%% Não se aplica
%%%% Texto
Dedico este trabalho a minha família e aos meus amigos, pelos momentos de ausência.
\end{Dedication}


%% Agradecimentos (elemento opcional; editar o {Arquivo} para alterar)
% %%%% AGRADECIMENTOS (ELEMENTO OPCIONAL)
%%
%% Texto (pessoal) em que se fazem agradecimentos dirigidos àqueles que
%% contribuíram de maneira relevante à elaboração do trabalho.
\begin{Acknowledgments}%[Título Alternativo]%% Substitui o título padrão
O presente trabalho não poderia ser finalizado sem a ajuda de diversas pessoas e/ou instituições às quais presto meus agradecimentos.
Certamente, esses parágrafos não abrangem todas as pessoas que fizeram parte dessa importante fase de minha vida.
Portanto, desde já peço desculpas àquelas que não estão presentes entre estas palavras, mas elas podem estar certas que fazem parte do meu pensamento com minha gratidão.\par%
A minha família, pelo carinho, incentivo e total apoio em todos os momentos da minha vida.\par%
Ao meu orientador, que me mostrou os caminhos a serem seguidos e pela confiança depositada.\par%
A todos os professores e colegas do curso, que ajudaram direta e indiretamente na realização e/ou conclusão deste trabalho.\par%
Aos demais que de alguma forma contribuíram para meu crescimento pessoal e profissional.\par%
%%%% À agência de fomento (último): {Nome}; [Número/Código de Fomento]
\FundingAgency{%
  da \ifbool{MakeAcr}{\intldescr{CAPES}}{Coordenação de Aperfeiçoamento de Pessoal de Nível Superior} \textemdash\ Brasil (\ifbool{MakeAcr}{\intl{CAPES}}{CAPES})%% CAPES
%   do \ifbool{MakeAcr}{\intl{CNPq}, \intldescr{CNPq}}{CNPq, Conselho Nacional de Desenvolvimento Científico e Tecnológico} \textemdash\ Brasil%% CNPq
%   da Fundação Araucária \textemdash\ Brasil%% FA
%   da \ifbool{MakeAcr}{\intl{UTFPR}, \intldescr{UTFPR}}{UTFPR, \UTFPRName} \textemdash\ Brasil%% UTFPR
}[%
  Código de Financiamento 001%% CAPES
%   \No\ de processo%% CNPq
%   \No\ de edital, financiamento, processo ou projeto%% FA
%   \No\ de edital, financiamento, processo ou projeto%% UTFPR
]
\end{Acknowledgments}


%% Epígrafe (elemento opcional; editar o {Arquivo} para alterar)
% %%%% EPÍGRAFE (ELEMENTO OPCIONAL)
%%
%% Texto em que se apresenta uma citação, seguida de indicação de autoria,
%% relacionada com a matéria tratada no corpo do trabalho.
%%%% Opção 1 (baseada na ABNT NBR 10520 - citações diretas curtas e longas):
%%%% - {Epigraph*} remove o formato de citação direta longa.
\begin{Epigraph}%% Argumentos (2):
% [0.5\textheight]%% Deslocamento vertical a partir da margem superior
% [Título]%% Não se aplica
%%%%%% Epígrafe(s) nos idiomas primário (texto) e original (nota de rodapé):
%%%%%% [Idioma] (brazilian ou english); {Autoria}; {Texto}; [Nota de Rodapé].
\Citation[brazilian]{\cite[tradução própria]{Einstein1921}}{%
  Até onde as leis da matemática se referem à realidade, não são certas; e até onde são certas, não se referem à realidade
}[%
  \Citation[english]{\cite{Einstein1921}}{%
    As far as the laws of mathematics refer to reality, they are not certain; and as far as they are certain, they do not refer to reality
  }.
].
\par%
\Citation[brazilian]{\cite[\ppno~37, tradução própria]{Asimov1950}}{%
  Primeira Lei: um robô não pode ferir um ser humano ou, por omissão, permitir que um ser humano sofra algum mal.
  Segunda Lei: um robô deve obedecer às ordens que lhe sejam dadas por seres humanos, exceto nos casos em que tais ordens contrariem a Primeira Lei.
  Terceira Lei: um robô deve proteger sua própria existência desde que tal proteção não entre em conflito com a Primeira ou Segunda Leis
}[%
  \Citation[english]{\cite[37]{Asimov1950}}{%
    First Law: a robot may not injure a human being or, through inaction, allow a human being to come to harm.
    Second Law: a robot must obey the orders given it by human beings except where such orders would conflict with the First Law.
    Third Law: a robot must protect its own existence as long as such protection does not conflict with the First or Second Laws
  }.
].
\end{Epigraph}
%%%% Opção 2 (baseada na classe de documento memoir)
% \begin{Epigraphs}%% Argumentos (2):
% [0.5\textheight]%% Deslocamento vertical a partir da margem superior
% [Título]%% Não se aplica
%%%%%% Epígrafe(s) nos idiomas primário (texto) e original (nota de rodapé):
%%%%%% [Idioma] (brazilian ou english); {Autoria}; {Texto}; [Nota de Rodapé].
% \EpiItem[brazilian]{\cite[tradução própria]{Einstein1921}}{%
%   Até onde as leis da matemática se referem à realidade, não são certas; e até onde são certas, não se referem à realidade.
% }[%
%   \Citation[english]{\cite{Einstein1921}}{%
%     As far as the laws of mathematics refer to reality, they are not certain; and as far as they are certain, they do not refer to reality
%   }.
% ]
% \end{Epigraphs}


%% Resumo
%% Apresentação concisa dos pontos relevantes de um texto, fornecendo uma visão
%% rápida e clara do conteúdo e das conclusões do trabalho:
%% - {Abstract*} insere a autorreferência do documento.
%%%% Chamada das Palavras-chave (opcional): {Estilo}
% \KeywordsCallFormat{\bfseries}%% Texto normal por padrão
%%%% Palavras-chave (de 3 a 6): {Número}; {Português}; {English}
% \Keyword{1}{palavra-chave 1}{keyword 1}
% \Keyword{2}{palavra-chave 2}{keyword 2}
% \Keyword{3}{palavra-chave 3}{keyword 3}
%%%% Contato (opção do pacote utfpr-thesis: Version = Abstract): {e-mail}
% \ContactEmail{student1@domain}
%%%% Em língua vernácula (idioma primário)
\begin{Abstract}%% Argumentos (2):
[brazilian]%% Idioma (brazilian ou english)
% [brazilian]%% Idioma da autorreferência do documento (brazilian ou english)
O resumo deve ser redigido na terceira pessoa do singular, com verbo na voz ativa, não ultrapassando uma página (de 150 a 500 palavras, conforme a norma vigente).
Evitar o uso de parágrafos no resumo, assim como abreviaturas, caracteres especiais, citações, equações, fórmulas e símbolos.
Iniciar o resumo situando o trabalho no contexto geral, apresentar os objetivos, descrever a metodologia adotada, relatar a contribuição, comentar os resultados obtidos e apresentar finalmente as conclusões mais importantes do trabalho.
As palavras-chave devem aparecer logo após o resumo, antecedidas da expressão \KeywordsCall, seguida de dois pontos, e separadas entre si por ponto e vírgula e finalizadas por ponto.
As palavras-chave devem ser grafadas com inicial minúscula, exceto nomes próprios ou científicos.
Por exemplo, \enquote{\KeywordsCall: gestação; Aedes aegypti; \ifbool{MakeAcr}{\intl{UTFPR}}{UTFPR}; Brasil.}.
Para definição das palavras-chave (e suas correspondentes em inglês no Abstract), consultar em Termo tópico do Catálogo de Autoridades da Biblioteca Nacional, disponível em: \url{https://acervo.bn.gov.br/sophia_web}.
Para editar o resumo e as palavras-chave, usar o arquivo \texttt{pre-textual.tex} em \texttt{./Pre-Textual/}.
\end{Abstract}
%%%% Em língua estrangeira (idioma secundário; para divulgação internacional)
\begin{Abstract}%% Argumentos (2):
[english]%% Idioma (brazilian ou english)
% [english]%% Idioma da autorreferência do documento (brazilian ou english)
The abstract should be drafted in the third-person singular with the verb in the active voice, at most one page (from 150 to 500 words, in accordance with the current regulation).
Avoid using paragraphs in the abstract, as well as abbreviations, special characters, quotes, equations, formulas, and symbols.
Initiate the abstract by setting the work in the general context, presenting the objectives, describing the methodology adopted, reporting the self-contribution, commenting on the results, and finally presenting the work's most relevant conclusions.
The keywords should appear after the abstract, preceded by the expression \KeywordsCall, followed by a colon, separated from each other by a semicolon, and ending with a period.
The keywords must be written with a lowercase initial, except for proper or scientific names.
For example, \enquote{\KeywordsCall: pregnancy; Aedes aegypti; \ifbool{MakeAcr}{\intl{UTFPR}}{UTFPR}; Brazil.}.
To define the keywords (and their corresponding Portuguese in the Resumo), query the Authorities Catalog Topic term in the National Library, available at: \url{https://acervo.bn.gov.br/sophia_web}.
To edit the abstract and keywords, use the file \texttt{pre-textual.tex} in \texttt{./Pre-Textual/}.
\end{Abstract}
