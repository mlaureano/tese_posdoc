%%%% AGRADECIMENTOS (ELEMENTO OPCIONAL)
%%
%% Texto (pessoal) em que se fazem agradecimentos dirigidos àqueles que
%% contribuíram de maneira relevante à elaboração do trabalho.
\begin{Acknowledgments}%[Título Alternativo]%% Substitui o título padrão
O presente trabalho não poderia ser finalizado sem a ajuda de diversas pessoas e/ou instituições às quais presto meus agradecimentos.
Certamente, esses parágrafos não abrangem todas as pessoas que fizeram parte dessa importante fase de minha vida.
Portanto, desde já peço desculpas àquelas que não estão presentes entre estas palavras, mas elas podem estar certas que fazem parte do meu pensamento com minha gratidão.\par%
A minha família, pelo carinho, incentivo e total apoio em todos os momentos da minha vida.\par%
Ao meu orientador, que me mostrou os caminhos a serem seguidos e pela confiança depositada.\par%
A todos os professores e colegas do curso, que ajudaram direta e indiretamente na realização e/ou conclusão deste trabalho.\par%
Aos demais que de alguma forma contribuíram para meu crescimento pessoal e profissional.\par%
%%%% À agência de fomento (último): {Nome}; [Número/Código de Fomento]
\FundingAgency{%
  da \ifbool{MakeAcr}{\intldescr{CAPES}}{Coordenação de Aperfeiçoamento de Pessoal de Nível Superior} \textemdash\ Brasil (\ifbool{MakeAcr}{\intl{CAPES}}{CAPES})%% CAPES
%   do \ifbool{MakeAcr}{\intl{CNPq}, \intldescr{CNPq}}{CNPq, Conselho Nacional de Desenvolvimento Científico e Tecnológico} \textemdash\ Brasil%% CNPq
%   da Fundação Araucária \textemdash\ Brasil%% FA
%   da \ifbool{MakeAcr}{\intl{UTFPR}, \intldescr{UTFPR}}{UTFPR, \UTFPRName} \textemdash\ Brasil%% UTFPR
}[%
  Código de Financiamento 001%% CAPES
%   \No\ de processo%% CNPq
%   \No\ de edital, financiamento, processo ou projeto%% FA
%   \No\ de edital, financiamento, processo ou projeto%% UTFPR
]
\end{Acknowledgments}
