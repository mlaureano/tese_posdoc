%%%% LISTA DE ABREVIATURAS E SIGLAS (ELEMENTO OPCIONAL; ENTRADAS)
%%
%% Consiste na relação alfabética das abreviaturas e siglas utilizadas no texto,
%% seguidas das palavras ou expressões correspondentes grafadas por extenso.
%% Recomenda-se a elaboração de lista própria para cada tipo.
%%
%% Notas:
%% 1. Sem subdivisões (\MakeAcronyms em ./utfpr-thesis.tex), os itens são
%%    ordenados pelo rótulo (label) em uma única lista.
%% 2. Com subdivisões (\MakeAcronyms* em ./utfpr-thesis.tex), os itens são
%%    agrupados por tipo e ordenados pelo rótulo (label).
%%
%% Definição de uma entrada:
%% \New<AcronymType>Entry{Label}{%% Usar somente letras latinas não acentuadas
%%   Term        = {...},%        % Obrigatório
%%   Description = {...},%        % Obrigatório
%%   Plural      = {...},%        % Opcional
%%   Translation = {...},%        % Opcional (descrição original em inglês)
%% }
%% Se a letra inicial do termo, da descrição ou do plural for acentuada, deve
%% ser colocada entre chaves, por exemplo, Description = {{á}rea}.
%%
%% Impressão no texto e adição (automática) na lista (exceto descrição):
%% +=========================+==================================+
%% | Comando                 | Imprime                          |
%% +-------------------------+----------------------------------+
%% | \<acrtype>{Label}       | Termo                            |
%% | \<AcrType>{Label}       | Termo (¹)                        |
%% | \<acrtype>descr{Label}  | Descrição                        |
%% | \<AcrType>Descr{Label}  | Descrição (¹)                    |
%% | \<acrtype>descr+{Label} | Descrição original em inglês     |
%% | \<AcrType>Descr+{Label} | Descrição original em inglês (¹) |
%% | \<acrtype>pl{Label}     | Plural                           |
%% | \<AcrType>Pl{Label}     | Plural (¹)                       |
%% +=========================+==================================+
%% (¹) Com letra inicial em maiúscula.
%%
%% Adição (automática) também no Índice Remissivo com um asterisco (*) opcional:
%% +==========================+==================================+
%% | Comando                  | Imprime                          |
%% +--------------------------+----------------------------------+
%% | \<acrtype>*{Label}       | Termo                            |
%% | \<AcrType>*{Label}       | Termo (¹)                        |
%% | \<acrtype>*descr{Label}  | Descrição                        |
%% | \<AcrType>*Descr{Label}  | Descrição (¹)                    |
%% | \<acrtype>descr+*{Label} | Descrição original em inglês     |
%% | \<AcrType>Descr+*{Label} | Descrição original em inglês (¹) |
%% | \<acrtype>pl*{Label}     | Plural                           |
%% | \<AcrType>Pl*{Label}     | Plural (¹)                       |
%% +==========================+==================================+
%% (¹) Com letra inicial em maiúscula.
%%
%% <AcronymType>, <acrtype> e <AcrType> são definidos conforme:
%% +=============+===============+===========+===========+
%% | Tipo        | <AcronymType> | <acrtype> | <AcrType> |
%% +-------------+---------------+-----------+-----------+
%% | Abreviatura | Abbreviation  | abrv      | Abrv      |
%% | Sigla       | Initials      | intl      | Intl      |
%% +=============+===============+===========+===========+

%% Definições de abreviaturas
\NewAbbreviationEntry{1D}{%
  Term        = {1D},%
  Description = {unidimensional},%
  Plural      = {1Ds},%
}
\NewAbbreviationEntry{art}{%
  Term        = {art.},%
  Description = {artigo},%
  Plural      = {arts.},%
}
\NewAbbreviationEntry{cap}{%
  Term        = {cap.},%
  Description = {capítulo},%
  Plural      = {caps.},%
}
\NewAbbreviationEntry{sec}{%
  Term        = {seç.},%
  Description = {seção},%
  Plural      = {seçs.},%
}

%% Definições de siglas
\NewInitialsEntry{ABNT}{%
  Term        = {ABNT},%
  Description = {Associação Brasileira de Normas Técnicas},%
}
\NewInitialsEntry{BMP}{%
  Term        = {BMP},%
  Description = {Mapa de Bits},%
  Translation = {Bitmap},%
}
\NewInitialsEntry{CAPES}{%
  Term        = {CAPES},%
  Description = {Coordenação de Aperfeiçoamento de Pessoal de Nível Superior},%
}
\NewInitialsEntry{CNPq}{%
  Term        = {CNPq},%
  Description = {Conselho Nacional de Desenvolvimento Científico e Tecnológico},%
}
\NewInitialsEntry{CTAN}{%
  Term        = {CTAN},%
  Description = {Comprehensive \TeX\ Archive Network},%
}
\NewInitialsEntry{EPS}{%
  Term        = {EPS},%
  Description = {Encapsulated PostScript},%
}
\NewInitialsEntry{GIF}{%
  Term        = {GIF},%
  Description = {Formato de Intercâmbio de Gráficos},%
  Translation = {Graphics Interchange Format},%
}
\NewInitialsEntry{GIMP}{%
  Term        = {GIMP},%
  Description = {Programa de Manipulação de Imagem \intl{GNU}},%
  Translation = {\intl{GNU} Image Manipulation Program},%
}
\NewInitialsEntry{GNU}{%
  Term        = {GNU},%
  Description = {GNU Não é Unix},%
  Translation = {GNU is Not Unix},%
}
\NewInitialsEntry{JPEG}{%
  Term        = {JPEG},%
  Description = {Joint Photographic Experts Group},%
}
\NewInitialsEntry{NBR}{%
  Term        = {NBR},%
  Description = {Norma Brasileira},%
}
\NewInitialsEntry{PDF}{%
  Term        = {PDF},%
  Description = {Formato de Documento Portátil},%
  Translation = {Portable Document Format},%
}
\NewInitialsEntry{PNG}{%
  Term        = {PNG},%
  Description = {Gráficos Portáteis de Rede},%
  Translation = {Portable Network Graphics},%
}
\NewInitialsEntry{PS}{%
  Term        = {PS},%
  Description = {PostScript},%
}
\NewInitialsEntry{QR}{%
  Term        = {QR},%
  Description = {Resposta Rápida},%
  Translation = {Quick Response},%
}
\NewInitialsEntry{TCC}{%
  Term        = {TCC},%
  Description = {Trabalho de Conclusão de Curso},%
}
\NewInitialsEntry{TUG}{%
  Term        = {TUG},%
  Description = {\TeX\ Users Group},%
}
\NewInitialsEntry{UML}{%
  Term        = {UML},%
  Description = {Linguagem de Modelagem Unificada},%
  Translation = {Unified Modeling Language},%
}
\NewInitialsEntry{URL}{%
  Term        = {URL},%
  Description = {Localizador Uniforme de Recursos},%
  Translation = {Uniform Resource Locator},%
}
\NewInitialsEntry{UTF8}{%
  Term        = {UTF-8},%
  Description = {Formato de Transformação Unicode de 8-bit},%
  Translation = {8{-}bit Unicode Transformation Format},%
}
\NewInitialsEntry{UTFPR}{%
  Term        = {UTFPR},%
  Description = {Universidade Tecnológica Federal do Paraná},%
}
