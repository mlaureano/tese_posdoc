\section{Inclusão de arquivos}%
\label{sect:files-incl}

Dividir o documento em diversos arquivos-fonte, ao invés de somente escrever tudo em um único, é uma prática bastante recomendável.
Esse recurso foi usado no documento, onde diversos arquivos são devidamente incluídos no principal arquivo-fonte (ver \texttt{./utfpr-thesis.tex}).

Para incluir diferentes arquivos em um arquivo-fonte (principal), de modo que cada arquivo incluído fique em página{(s)} distinta{(s)}, ou seja, com quebra de páginas, como em seções primárias, usa-se o comando:

\begin{snugshade}
\begin{Verbatim}
\include{file-to-include}%% Sem a extensão tex
\end{Verbatim}
\end{snugshade}

Para controlar quais arquivos são lidos pelo \ifbool{MakeGly}{\gly*{LaTeX}}{\LaTeX} nos comandos \ColorBox{shadecolor}{\texttt{{\textbackslash}include}} subsequentes, usa-se o seguinte comando no preâmbulo do documento:

\begin{snugshade}
\begin{Verbatim}
\includeonly{%
  file-1-to-include,%% Arquivo 1 (sem a extensão tex)
  file-2-to-include,%% Arquivo 2 (sem a extensão tex)
  file-3-to-include,%% Arquivo 3 (sem a extensão tex)
}
\end{Verbatim}
\end{snugshade}

\noindent Onde no argumento deste comando são adicionados os nomes de arquivos separados por vírgulas.

Para incluir arquivos sem quebra de páginas, como em seções secundárias, usa-se o comando:

\begin{snugshade}
\begin{Verbatim}
\input{file-to-include}%% Sem a extensão tex
\end{Verbatim}
\end{snugshade}
