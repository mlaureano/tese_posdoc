\section{Enumerações: alíneas e subalíneas}%
\label{sect:enum}%
\index{alínea}%
\index{alínea!subalínea}

Quando for necessário enumerar os diversos assuntos que não possuam título próprio e estão contidos em uma seção, estes devem ser subdivididos em alíneas\index{alínea}~\cite[\ifbool{MakeAcr}{\Abrv{sec}}{Seç.} 4.2 \ppno~3 da \ifbool{MakeAcr}{\intl{NBR}}{NBR} 6024]{ABNT2012NBR6024}:

\begin{enumerate}[label = {\alphsect}]
\item o texto que antecede as alíneas\index{alínea} termina em dois pontos;
\item as alíneas\index{alínea} devem ser indicadas alfabeticamente, em letra minúscula, seguida de parêntese; utilizam-se letras dobradas, quando esgotadas as letras do alfabeto;
\item as letras indicativas das alíneas\index{alínea} devem apresentar recuo em relação à margem esquerda;
\item o texto da \Index{alínea} deve começar por letra minúscula e terminar em ponto-e-vírgula, exceto a última \Index{alínea}, que deve terminar em ponto final;
\item o texto da \Index{alínea} deve terminar em dois pontos, se houver \Index[alínea]{subalínea};
\item a segunda e as seguintes linhas do texto da \Index{alínea} começam sob a primeira letra do texto da própria \Index{alínea};
\item as subalíneas\index{alínea!subalínea}~\cite[\ifbool{MakeAcr}{\Abrv{sec}}{Seç.} 4.3 \ppno~4 da \ifbool{MakeAcr}{\intl{NBR}}{NBR} 6024]{ABNT2012NBR6024} devem ser elaboradas conforme as alíneas\index{alínea} a seguir:
\begin{enumerate}[label = {\textendash}]
\item as subalíneas\index{alínea!subalínea} devem começar por travessão seguido de espaço;
\item as subalíneas\index{alínea!subalínea} devem apresentar recuo em relação à \Index{alínea};
\item o texto da \Index[alínea]{subalínea} deve começar por letra minúscula e terminar em ponto-e-vírgula; a última \Index[alínea]{subalínea} deve terminar em ponto final, se não houver \Index{alínea} subsequente;
\item a segunda e as seguintes linhas do texto da \Index[alínea]{subalínea} começam sob a primeira letra do texto da própria \Index[alínea]{subalínea};
\end{enumerate}
\item \textbf{\Index{alínea} em negrito}:
\begin{enumerate}[label = {\textendash}]
\item \textit{\Index[alínea]{subalínea} em itálico};
\item \uline{\textit{\Index[alínea]{subalínea} em itálico e sublinhado}};
\end{enumerate}
\item última \Index{alínea} contendo uma palavra com \emph{ênfase}.
\end{enumerate}
