\section{Glossário}%
\label{sect:gly}

Termos de \glyref\ podem ser definidos ao longo do texto, preferencialmente antes do seu primeiro uso, ou em um arquivo de entradas.
Tal arquivo pode ser incluído no preâmbulo do principal arquivo-fonte (ver \texttt{./utfpr-thesis.tex}) pelo comando:

\begin{snugshade}
\begin{Verbatim}
\MakeGlossary[File]%% Por exemplo, glossary-entries.tex em ./Post-Textual/Optionals/
\end{Verbatim}
\end{snugshade}

Os termos são definidos usando o comando do pacote \texttt{utfpr-thesis}:

\begin{snugshade}
\begin{Verbatim}
\NewGlossaryEntry{Label}{%% Usar somente letras latinas não acentuadas
  Term        = {...},%   % Obrigatório
  Description = {...},%   % Obrigatório
  Plural      = {...},%   % Opcional
  Parent      = {...},%   % Opcional
}
\end{Verbatim}
\end{snugshade}

\noindent Observação: se a letra inicial do termo, da descrição ou do plural for acentuada, deve ser colocada entre chaves, por exemplo, \ColorBox{shadecolor}{\texttt{Description = \{\{á\}rea\}}}.
Isto se aplica também para abreviaturas e siglas e para símbolos (exceto termo e plural para este último).

Para que termos de \glyref\ sejam impressos em alguma parte do texto do documento (e, consequentemente, sejam adicionados no \glyref), usam-se os comandos do pacote \texttt{utfpr-thesis}:

\begin{snugshade}
\begin{Verbatim}
\gly{Label}%     % Termo
\Gly{Label}%     % Termo com letra inicial em maiúscula
\glydescr{Label}%% Descrição
\GlyDescr{Label}%% Descrição com letra inicial em maiúscula
\glypl{Label}%   % Plural
\GlyPl{Label}%   % Plural com letra inicial em maiúscula
\end{Verbatim}
\end{snugshade}

\noindent Os comandos com letra inicial em maiúscula imprimem o termo, a descrição e o plural (se definido) com a letra inicial em maiúscula.
As versões destes comandos com um asterisco (*) opcional, por exemplo, \ColorBox{shadecolor}{\Verb|\gly*{Label}|}, adicionam o termo também no \idxref, exceto os comandos de descrição.
Seguem alguns exemplos.

\enquote{\ifbool{MakeGly}{\gly*{UTFPRThesis} é um \glydescr{UTFPRThesis}}{\UTFPR-Thesis é um modelo \LaTeX\ que permite atender os requisitos das normas definidas pela UTFPR para elaboração de trabalhos acadêmicos}}.
Esta citação direta curta corresponde a um exemplo de termo definido no \glyref\ e usado no decorrer do texto, assim como:

\begin{DisplayCitation}[brazilian]{}
Esta frase usa a palavra \ifbool{MakeGly}{\gly*{componente}}{componente} e o plural \ifbool{MakeGly}{\glypl*{filho}}{filhos}, ambas definidas no \glyref\ como filhas da entrada \ifbool{MakeGly}{\gly*{pai}}{pai}.
\ifbool{MakeGly}{\Gly*{equilibriodaconfiguracao}}{Equilíbrio da configuração} exemplifica o uso de um termo no início de uma frase.
O modelo \ifbool{MakeGly}{\gly*{UTFPRThesis}}{\UTFPR-Thesis} é escrito em \ifbool{MakeGly}{\gly*{LaTeX}}{\LaTeX}, definido no \glyref\ como \enquote{\ifbool{MakeGly}{\glydescr{LaTeX}}{conjunto de macros para o processador de textos \TeX, utilizado amplamente para a produção de textos matemáticos e científicos devido à sua alta qualidade tipográfica}}.
\end{DisplayCitation}

O texto desta \Index{citação} \Index[citação]{direta} longa foi produzido com:

\begin{snugshade}
\begin{Verbatim}[numbers = left]
\begin{DisplayCitation}[brazilian]{}
Esta frase usa a palavra \ifbool{MakeGly}{\gly*{componente}}{componente} e o plural \ifbool{MakeGly}{\glypl*{filho}}{filhos}, ambas definidas no \glyref\ como filhas da entrada \ifbool{MakeGly}{\gly*{pai}}{pai}.
\ifbool{MakeGly}{\Gly*{equilibriodaconfiguracao}}{Equilíbrio da configuração} exemplifica o uso de um termo no início de uma frase.
O modelo \ifbool{MakeGly}{\gly*{UTFPRThesis}}{\UTFPR-Thesis} é escrito em \ifbool{MakeGly}{\gly*{LaTeX}}{\LaTeX}, definido no \glyref\ como \enquote{\ifbool{MakeGly}{\glydescr{LaTeX}}{conjunto de macros para o processador de textos \TeX, utilizado amplamente para a produção de textos matemáticos e científicos devido à sua alta qualidade tipográfica}}.
\end{DisplayCitation}
\end{Verbatim}
\end{snugshade}
