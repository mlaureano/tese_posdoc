\section{Equações}%
\label{sect:eq}

\ifbool{MakeGly}{\gly*{LaTeX}}{\LaTeX} é insuperável no processamento de equações.
Símbolos ou expressões matemáticas simples podem ser inseridos ao longo do texto de um parágrafo usando o ambiente \texttt{math} (ver \Cref{sect:math}) ou os comandos para impressão de símbolos definidos na \Cref{sect:sym}.
Por exemplo: $\nu = \mu / \rho$, sendo \ifbool{MakeSym}{\grkl{nu} a \grkldescr{nu}, \grkl{mu} a \grkldescr{mu} e \grkl{rho} a \grkldescr{rho}}{$\nu$ a viscosidade cinemática, $\mu$ a viscosidade dinâmica e $\rho$ a massa específica}.

Equações simples como $A = \pi D^2 / 4$ podem ser adicionadas ao longo do texto de um parágrafo ou em uma linha própria usando o ambiente \texttt{displaymath}:
\begin{displaymath}
A = \frac{\pi D^2}{4} \equiv \pi R^2
\end{displaymath}

\noindent Sendo \ifbool{MakeSym}{\ltnl{A} a \ltnldescr{A}, \grkl{pi} a \grkldescr{pi}, \ltnl{D} ($\equiv 2 R$) o \ltnldescr{D} e \ltnl{R} o \ltnldescr{R}}{$A$ a área, $\pi$ a constante circular (Pi), $D$ ($\equiv 2 R$) o diâmetro e $R$ o raio}.

Por outro lado, o ambiente \texttt{equation} pode ser usado para gerar equações então numeradas automaticamente e podem ser referenciadas ao longo do texto.
Por exemplo, a \Cref{eq:p-gamma} é trivialmente derivada da \Cref{eq:T-r}:
\begin{equation}%
\label{eq:p-gamma}
\begin{array}{lcl}
p \left(\gamma\right)
& = &
\frac{1}{2}
\sqrt{\frac{M}{\gamma \bar{\gamma}_b}}
\frac{1}{\prod_{i = 1}^M \sqrt{\tilde{\gamma}_i}}
\int_0^{\sqrt{M \delta}}
\int_0^{\sqrt{M \delta} - r_M} \cdots
\int_0^{\sqrt{M \delta} - \sum_{i = 3}^M r_i} \\[0.5\onelineskip]
& &
p \left(%
  \frac{\sqrt{M \delta} - \sum_{i = 2}^M r_i}{\sqrt{\tilde{\gamma}_1}},
  \frac{r_2}{\sqrt{\tilde{\gamma}_2}}, \ldots,
  \frac{r_M}{\sqrt{\tilde{\gamma}_M}}
\right) \, \mathrm{d} r_2 \cdots \mathrm{d} r_{M - 1} \, \mathrm{d} r_M
\end{array}
\end{equation}
\begin{equation}%
\label{eq:T-r}
T \left(r\right) =
\frac{1}{f_m}
{\left(%
  \frac{\pi}{2} \sum_{i = 1}^M {\tilde{r}_i^2 \dot{\varsigma}_i^2}
\right)}^{-1/2}
\frac{%
  \begin{array}{l}
  \int_0^{\rho \sqrt{M}}
  \int_0^{\rho \sqrt{M} - r_M} \cdots
  \int_0^{\rho \sqrt{M} - \sum_{i = 3}^M r_i}
  \int_0^{\rho \sqrt{M} - \sum_{i = 2}^M r_i} \\[0.5\onelineskip]
  p \left(%
    \frac{r_1}{\tilde{r}_1},
    \frac{r_2}{\tilde{r}_2}, \ldots,
    \frac{r_M}{\tilde{r}_M}
  \right) \, \mathrm{d} r_1 \, \mathrm{d} r_2 \cdots \mathrm{d} r_{M - 1} \, \mathrm{d} r_M \\[0.5\onelineskip]
  \end{array}
}{%
  \begin{array}{l}
  \int_0^{\rho \sqrt{M}}
  \int_0^{\rho \sqrt{M} - r_M} \cdots
  \int_0^{\rho \sqrt{M} - \sum_{i = 3}^M r_i} \\[0.5\onelineskip]
  p \left(%
    \frac{\rho \sqrt{M} - \sum_{i = 2}^M r_i}{\tilde{r}_1},
    \frac{r_2}{\tilde{r}_2}, \ldots,
    \frac{r_M}{\tilde{r}_M}
  \right) \, \mathrm{d} r_2 \cdots \mathrm{d} r_{M - 1} \, \mathrm{d} r_M \\[0.5\onelineskip]
  \end{array}
}
\end{equation}

Diversas ferramentas podem ser usadas para gerar ou editar equações, por exemplo: \ENLang{\href{https://formulasheet.com/}{Formula Sheet\LinkIcon} e \href{https://www.tutorialspoint.com/latex_equation_editor.htm}{\ifbool{MakeGly}{\gly*{LaTeX}}{\LaTeX} Equation Editor (\textit{by} Tutorials Point)\LinkIcon}}.
