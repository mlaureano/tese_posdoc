%%%% CAPÍTULO 2: REVISÃO DA LITERATURA (OU FUNDAMENTAÇÃO TEÓRICA; OU ESTADO DA
%%%% ARTE; OU ESTADO DO CONHECIMENTO)
%%
%% Deve ser registrado o conhecimento sobre a literatura essencial do assunto,
%% discutindo e comentando a informação já publicada. A revisão deve ser
%% apresentada, preferencialmente, por blocos de assunto e em ordem cronológica,
%% procurando mostrar a evolução do tema.

%% Locais (pastas) de ilustrações deste capítulo
\graphicspath{%
  {./Chapter-2/}%% Primário
%   {./Chapter-2/Illustrations/}%% Secundário (descomentar se houver)
}

\chapter{Revisão da Literatura}%
\label{chpt:lit-rev}

Deve ser registrado o conhecimento sobre a literatura essencial do assunto, discutindo e comentando a informação já publicada.
A revisão deve ser apresentada, preferencialmente, por blocos de assunto e em ordem cronológica, procurando mostrar a evolução do tema.

%% Notas:
%% 1. O capítulo de exemplo exibe informações e exemplos sobre comandos e
%%    ambientes TeX/LaTeX, assim como os específicos do pacote utfpr-thesis,
%%    entre outros. Estes comandos e ambientes podem ser utilizados para
%%    produzir as diversas estruturas necessárias à elaboração do documento:
%%    enumerações; citações e referências; equações; ilustrações; tabelas;
%%    abreviaturas, siglas e acrônimos; símbolos; glossário; apêndices e anexos;
%%    índice; entre outras. Observar comentários em todos os arquivos-fonte do
%%    projeto para entender configurações e formatações no modelo.
%% 2. O arquivo final (PDF) pode ser convertido para PDF/A usando diversas
%%    ferramentas, por exemplo:
%%      https://www.pdfforge.org/online/en/pdf-to-pdfa
