%%%% GLOSSÁRIO (ELEMENTO OPCIONAL; ENTRADAS)
%%
%% Relação (alfabética) de palavras ou expressões técnicas de uso restrito, ou
%% de sentido obscuro, utilizadas no texto, acompanhadas das respectivas
%% definições.
%%
%% Definição de uma entrada:
%% \NewGlossaryEntry{Label}{%% Usar somente letras latinas não acentuadas
%%   Term        = {...},%   % Obrigatório
%%   Description = {...},%   % Obrigatório
%%   Plural      = {...},%   % Opcional
%%   Parent      = {...},%   % Opcional
%% }
%% Se a letra inicial do termo, da descrição ou do plural for acentuada, deve
%% ser colocada entre chaves, por exemplo, Description = {{á}rea}.
%%
%% Impressão no texto e adição (automática) no Glossário (exceto descrição):
%% +==================+===============+
%% | Comando          | Imprime       |
%% +------------------+---------------+
%% | \gly{Label}      | Termo         |
%% | \Gly{Label}      | Termo (¹)     |
%% | \glydescr{Label} | Descrição     |
%% | \GlyDescr{Label} | Descrição (¹) |
%% | \glypl{Label}    | Plural        |
%% | \GlyPl{Label}    | Plural (¹)    |
%% +==================+===============+
%% (¹) Com letra inicial em maiúscula.
%%
%% Adição (automática) também no Índice Remissivo com um asterisco (*) opcional:
%% +================+============+
%% | Comando        | Imprime    |
%% +----------------+------------+
%% | \gly*{Label}   | Termo      |
%% | \Gly*{Label}   | Termo (¹)  |
%% | \glypl*{Label} | Plural     |
%% | \GlyPl*{Label} | Plural (¹) |
%% +================+============+
%% (¹) Com letra inicial em maiúscula.

%% Definições de itens de Glossário
\NewGlossaryEntry{biber}{%
  Term        = {biber},%
  Description = {substituto do \gly*{BibTeX} para usuários do \gly*{BibLaTeX}},%
  Parent      = {LaTeX},%
}
\NewGlossaryEntry{BibLaTeX}{%
  Term        = {Bib\LaTeX},%
  Description = {reimplementação completa das facilidades bibliográficas fornecidas pelo \gly*{LaTeX}},%
  Parent      = {LaTeX},%
}
\NewGlossaryEntry{BibLaTeXabnt}{%
  Term        = {Bib\LaTeX-abnt},%
  Description = {pacote que oferece um estilo \gly*{BibLaTeX} que atende às regras da \ifbool{MakeAcr}{\intl{ABNT}}{ABNT}},%
  Parent      = {LaTeX},%
}
\NewGlossaryEntry{BibTeX}{%
  Term        = {Bib\TeX},%
  Description = {aplicativo de gerenciamento de referências para a formatação de listas de referências no \gly*{LaTeX}},%
  Parent      = {LaTeX},%
}
\NewGlossaryEntry{componente}{%
  Term        = {componente},%
  Description = {outro exemplo de uma entrada secundária (componente), subentrada da primária chamada \gly*{pai}; trata-se de uma entrada irmã de outra também secundária chamada \gly*{filho}},%
  Plural      = {componentes},%
  Parent      = {pai},%
}
\NewGlossaryEntry{dissertacao}{%
  Term        = {dissertação},%
  Description = {trabalho acadêmico desenvolvido no mestrado},%
  Plural      = {dissertações},%
}
\NewGlossaryEntry{equilibriodaconfiguracao}{%
  Term        = {equilíbrio da configuração},%
  Description = {consistência entre os \glypl*{componente}},%
  Plural      = {equilíbrios das configurações},%
}
\NewGlossaryEntry{filho}{%
  Term        = {filho},%
  Description = {exemplo de uma entrada secundária (filho), subentrada da primária chamada \gly*{pai}},%
  Plural      = {filhos},%
  Parent      = {pai},%
}
\NewGlossaryEntry{LaTeX}{%
  Term        = {\LaTeX},%
  Description = {conjunto de macros para o processador de textos \gly*{TeX}, utilizado amplamente para a produção de textos matemáticos e científicos devido à sua alta qualidade tipográfica},%
  Parent      = {TeX},%
}
\NewGlossaryEntry{memoir}{%
  Term        = {memoir},%
  Description = {classe \gly*{LaTeX} que permite a composição de poesia, ficção, obras de não ficção e matemáticas, como livros, relatórios, \ifbool{MakeAcr}{\abrvdescr*{art}s}{artigos} ou manuscritos},%
  Parent      = {LaTeX},%
}
\NewGlossaryEntry{pai}{%
  Term        = {pai},%
  Description = {exemplo de entrada primária (pai) que possui subentradas ou entradas secundárias (\glypl*{filho})},%
  Plural      = {pais},%
}
\NewGlossaryEntry{tese}{%
  Term        = {tese},%
  Description = {trabalho acadêmico desenvolvido no doutorado},%
  Plural      = {teses},%
}
\NewGlossaryEntry{TeX}{%
  Term        = {\TeX},%
  Description = {sistema de tipografia criado por Donald E. Knuth},%
}
\NewGlossaryEntry{UTFPRThesis}{%
  Term        = {\UTFPR-Thesis},%
  Description = {modelo \gly*{LaTeX} que permite atender os requisitos das normas definidas pela \ifbool{MakeAcr}{\intl{UTFPR}}{UTFPR} para elaboração de trabalhos acadêmicos},%
  Parent      = {LaTeX},%
}
