%%%% CAPÍTULO 5: CONCLUSÕES (OU CONSIDERAÇÕES FINAIS)
%%
%% Deve finalizar o trabalho com respostas às hipóteses especificadas na
%% introdução. O ponto de vista sobre os resultados obtidos deve ser expresso;
%% não se deve incluir novos dados ou equações. A partir da tese, alguns
%% assuntos identificados como importantes para serem explorados podem ser
%% sugeridos como temas para novas pesquisas.

%% Locais (pastas) de ilustrações deste capítulo
\graphicspath{%
  {./Chapter-5/}%% Primário
%   {./Chapter-5/Illustrations/}%% Secundário (descomentar se houver)
}

\chapter{Conclusões}%
\label{chpt:concl}

Deve finalizar o trabalho com respostas às hipóteses especificadas na introdução.
O ponto de vista sobre os resultados obtidos deve ser expresso; não se deve incluir novos dados ou equações.
A partir da tese, alguns assuntos identificados como importantes para serem explorados podem ser sugeridos como temas para novas pesquisas.

%% Notas:
%% 1. O capítulo de exemplo exibe informações e exemplos sobre comandos e
%%    ambientes TeX/LaTeX, assim como os específicos do pacote utfpr-thesis,
%%    entre outros. Estes comandos e ambientes podem ser utilizados para
%%    produzir as diversas estruturas necessárias à elaboração do documento:
%%    enumerações; citações e referências; equações; ilustrações; tabelas;
%%    abreviaturas, siglas e acrônimos; símbolos; glossário; apêndices e anexos;
%%    índice; entre outras. Observar comentários em todos os arquivos-fonte do
%%    projeto para entender configurações e formatações no modelo.
%% 2. O arquivo final (PDF) pode ser convertido para PDF/A usando diversas
%%    ferramentas, por exemplo:
%%      https://www.pdfforge.org/online/en/pdf-to-pdfa
