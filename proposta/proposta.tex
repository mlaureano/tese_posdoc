%%%% utfpr-article.tex, 2025/08/07
%%%% Copyright (C) 2016-2025 Luiz E. M. Lima (luizeduardomlima@gmail.com)
%%
%% This work may be distributed and/or modified under the conditions of the
%% LaTeX Project Public License, either version 1.3 of this license or (at your
%% option) any later version.
%% The latest version of this license is in
%%   https://www.latex-project.org/lppl.txt
%% and version 1.3 or later is part of all distributions of LaTeX version
%% 2005/12/01 or later.
%%
%% This work has the LPPL maintenance status `maintained'.
%%
%% The Current Maintainer of this work is Luiz E. M. Lima.
%%
%% This work consists of all files listed in readme.html.
%%
%% A brief description of this work is in readme.txt.

%% Conformidade com PDF/A (LaTeX release 2022-06-01 or later)
% \DocumentMetadata{pdfstandard = A-3b, pdfversion = 1.7, lang = pt-BR}

%% Detecção e aviso sobre comandos obsoletos
% \RequirePackage[l2tabu, orthodox]{nag}

%% Classe de documento
\documentclass[%% Opções (^ = padrão; > = para pacotes):
  a4paper,%% Tamanho de papel: a4paper, letterpaper (^), etc.
  12pt,%% Tamanho de fonte: 10pt (^), 11pt ou 12pt
  fleqn,%% Alinhamento de equações à esquerda (centralizado por padrão)
%   draft,%% Aparência de documento (>): draft ou final (^)
  english,%% Idioma secundário (penúltimo) (>)
  brazilian,%% Idioma primário (último) (>)
]{article}

%% Pacotes utilizados
\usepackage[%% Opções (^ = padrão; ¹ = Leiaute A; ² = Leiaute B):
%   Layout    = A,%% Leiaute (margens): A (esquerda menor + recuo) ou B (iguais) (^)
%   Font      = Times,%% Fonte principal: Arial (²), Calibri (¹), Times ou LaTeX(*)
%   TextColor = 575757,%% Cor (HTML) texto (preto ^): cinzento
%   EmphColor = 008080,%% Cor (HTML) ênfase (preto ^): verde azulado
%   LinkColor =,%% Cor (HTML) hyperlink (azul escuro ^): nenhuma (cor texto)
%   Rule      = Both,%% Linha (cabeçalho e rodapé): None (²), Header, Footer ou Both (¹)
%   AffilIcon = Off,%% Ícones de e-mail, Lattes, ORCID e Instituição: On (^) ou Off
%   DocLic    = None,%% Licença do documento: CC (Creative Commons) (^) ou None
%   CCLic     = BY-NC-ND,%% Licença CC: BY (^), BY-SA, BY-ND, BY-NC, BY-NC-SA ou BY-NC-ND
%   PageNum   = Off,%% Numeração de páginas: On (^) ou Off
%   SectNum   = Off,%% Numeração de seções (\section): On (²) ou Off (¹)
%   UnnumSect = Center,%% Alinhamento de seções (\section*): Center (¹) ou Left (²)
%   Caption   = Left,%% Alinhamento de legendas: Center (^) ou Left
%   Source    = Left,%% Alinhamento de fontes (autoria): Center (^) ou Left
%   ABNTCit   = NSB,%% Citação ABNT: AAY (Nome, Ano) (^), NRB \cite{meegleYouthRobotics2025} ou NSB [1]
%   DOIIcon   = On,%% Ícone de DOI em referências: On ou Off (^)
%   URLIcon   = On,%% Ícone de URL em referências: On ou Off (^)
%   Index     = On,%% Índice Remissivo (makeindex): On ou Off (^)
]{utfpr-article}

%% Informações do documento (descomentar para alterar; * = opcional)
%%%% Título: [Cabeçalho]; {Português}; {English}
 \Title{%
   \ifbool{@LayoutA}{%
     Aplicação de Braços Robóticos para Inspirar Jovens na Pesquisa Robótica: Uso em Ambientes Educacionais Informais%
   }{%
    Aplicação de Braços Robóticos para Inspirar Jovens na Pesquisa Robótica: Uso em Ambientes Educacionais Informais%
   }%
 }{%
   \ifbool{@LayoutA}{%
     Using Robotic Arms to Inspire Young People in Robotics Research: A Case for Informal Educational Settings%
   }{%
     Using Robotic Arms to Inspire Young People in Robotics Research: A Case for Informal Educational Settings%
   }%
 }
%%%% Autor(es) (de 1 a 8, Leiaute A; de 1 a 16, Leiaute B): {Número}; {Dados}
 \Author{1}{%% Bolsista ou voluntário(a) principal (primeiro)
   Name        = {Marcos Aurelio Pchek Laureano},%
   Email       = {marcos.laureano@ifpr.edu.br},%
%   Lattes      = {0000000000000001},%% (*)
%   ORCID       = {0000-0000-0000-0001},%% (*) (CHKTEX 8)
   Affiliation = {\UTFPRName, Curitiba, Paraná, Brasil},%
%   Role        = {Bolsista Modalidade/Agência},%% Leiaute B
 }
% \Author{2}{%% Colaborador(a) (segundo ao penúltimo)
%   Name        = {Nome Completo do{(a)} Autor{(a)}-A2},%
%   Email       = {author2@domain},%
%   Lattes      = {0000000000000002},%% (*)
%   ORCID       = {0000-0000-0000-0002},%% (*) (CHKTEX 8)
%   Affiliation = {Instituição (Nome por Extenso), Cidade, Estado, País},%
%   Role        = {Discente de Nome do Curso},%% Leiaute B
% }
% \Author{3}{%% Orientador(a) principal (último)
%   Name        = {Nome Completo do{(a)} Autor{(a)}-A3},%
%   Email       = {author3@domain},%
%   Lattes      = {0000000000000003},%% (*)
%   ORCID       = {0000-0000-0000-0003},%% (*) (CHKTEX 8)
%   Affiliation = {\UTFPRName, Cidade, Paraná, Brasil},%
%   Role        = {Docente do Nome do Departamento ou Programa},%% Leiaute B
% }
%%%%%% Autor(es) (lista curta) (*): {Cabeçalho}
% \Authors{N.~C.~Autor{(a)}-A1 \textit{et al}.}
%%%% Digital Object Identifier (DOI) (*): {Name}
% \DOIName{10.1000/xyz123}
%%%% Datas (*): [Ano] (atual por padrão); {Recebido}; {Aprovado}
% \Dates[2025]{DD mmm. \YearNum}{DD mmm. \YearNum}
%%%% Atribuição de Licença CC (*): {Ano}; {Autoria}
% \CCLicenseAttribution{\YearNum}{N.~C.~Autor{(a)}-A1 \textit{et al}.}
%%%% Evento (SEI, SICITE, etc.): {Dados}
% \EventInfo{%% \EventInfo* faz cabeçalhos iguais (imagens e dados)
%   Name     = {Nome Completo do Evento},%
%   Acronym  = {EVNT \YearNum},%% (*)
%   Date     = {DD a DD de Mmmmmm de \YearNum},%
%   Local    = {Cidade, Paraná, Brasil},%
%   Logo     = {example-image},%% Arquivo de imagem (*) em ./Logos/
%   Header   = {example-image},%% Arquivo de imagem (*) em ./Headers/
%   HdrWidth = {text},%% Largura de linha e/ou imagem (Header): paper (padrão) ou text
%   URL      = {https://www.example.com/},%
% }
%%%% Instituição (UTFPR se nenhum evento): {Dados}
% \InstitutionInfo{%% \InstitutionInfo* faz cabeçalhos iguais (imagens e dados)
%   Campus     = {Cidade},%
%   Department = {Departamento, Coordenação, Programa ou Curso},%
%   DeptLogo   = {example-image},%% Arquivo de imagem (*) em ./Logos/
% }
%%%% Objetivos de Desenvolvimento Sustentável (ODS) (de 1 a 3 arquivos de imagem
%%%% em ./SGDStamps/) (*): {Selo-1}; {Selo-2}; {Selo-3}
% \SDGStamps{example-image}{example-image}{example-image}

%% Resumo e Palavras-chave (* = opcional)
%%%% Margens (recuo a partir das margens do texto; Leiaute B) (*)
% \setlength{\AbstractMargin}{10mm}%% 0mm por padrão
%%%% Chamada do Resumo (*): {Estilo}
% \AbstractCallFormat{\bfseries}%% Destacado (caixa alta e negrito por padrão)
% \AbstractCallFormat*{\bfseries}%% Em linha (caixa alta e negrito por padrão)
%%%% Chamada das Palavras-chave (*): {Estilo}
% \KeywordsCallFormat{\bfseries}%% Caixa alta e negrito por padrão
%%%% Palavras-chave (de 3 a 5): {Número}; {Português}; {English}
\Keyword{1}{braço robótico}{robotic arm}
\Keyword{2}{STEAM}{STEAM}
%\Keyword{3}{palavra chave-3}{keyword 3}
%%%% Resumo no idioma primário (brazilian)
\begin{Abstract}[brazilian]%% Idioma (brazilian ou english)
Este artigo explora a aplicação estratégica de braços robóticos em feiras de tecnologia e mostras científicas como uma ferramenta para motivar, incentivar e despertar o interesse de crianças e adolescentes na pesquisa robótica. A análise abrangente, fundamentada em uma revisão bibliográfica extensiva, demonstra que a robótica educacional em ambientes informais não apenas aumenta o engajamento em Ciência, Tecnologia, Engenharia, Artes e Matemática (STEAM), mas também cultiva habilidades essenciais para o século XXI, como resolução de problemas, pensamento crítico, criatividade e colaboração.
O trabalho detalha as características ideais para a seleção de braços robóticos, a prototipagem de cenários interativos e a integração de conceitos de Inteligência Artificial (IA) e Aprendizado de Máquina. Apresenta metodologias para a avaliação do engajamento e a análise de dados, enfatizando uma abordagem de métodos mistos para capturar tanto as mudanças quantificáveis quanto as experiências qualitativas dos participantes. Além disso, são documentadas práticas e propostas diretrizes para garantir a segurança, a eficácia e a inclusão em programas de extensão em robótica, incluindo considerações éticas na interação humano-robô. A implementação estratégica dessas recomendações pode estabelecer pontes significativas entre a aprendizagem informal e os caminhos formais para a pesquisa e carreiras em STEAM.
\end{Abstract}
%%%% Resumo no idioma secundário (english)
\begin{Abstract}[english]%% Idioma (brazilian ou english)
This paper explores the strategic application of robotic arms at technology fairs and science exhibitions as a tool to motivate, encourage, and spark the interest of children and adolescents in robotics research. A comprehensive analysis, grounded in an extensive literature review, demonstrates that educational robotics in informal settings not only increases engagement in Science, Technology, Engineering, Arts, and Mathematics (STEAM), but also cultivates essential 21st-century skills such as problem-solving, critical thinking, creativity, and collaboration.
The study details the ideal characteristics for selecting robotic arms, the prototyping of interactive scenarios, and the integration of Artificial Intelligence (AI) and Machine Learning concepts. It presents methodologies for evaluating engagement and analyzing data, emphasizing a mixed-methods approach to capture both quantifiable changes and the qualitative experiences of participants. Furthermore, it documents best practices and proposes guidelines to ensure the safety, effectiveness, and inclusivity of robotics outreach programs, including ethical considerations in human-robot interaction. The strategic implementation of these recommendations can establish meaningful bridges between informal learning and formal pathways to STEAM research and careers.
\end{Abstract}

%% Legendas e fontes (autoria) de ilustrações e tabelas em negrito
\BoldCaption%% Texto normal por padrão

%% Processamento (makeindex) de entradas (itens) do Índice Remissivo
\makeindex%

%% Arquivo(s) de referências
\addbibresource{atualizado_completo_normalizado.bib}

%% Início do documento
\begin{document}

%% Elementos pré-textuais (opcionais)
%% Editar o {Arquivo} para alterar: Lista de Ilustrações, Lista de Tabelas,
%% Lista de Abreviaturas e Siglas, Lista de Símbolos e Sumário.
% %%%% Elementos pré-textuais (opcionais)
%%
%% Lista de Ilustrações, Lista de Tabelas, Lista de Abreviaturas e Siglas, Lista
%% de Símbolos e Sumário.
%%
%% Tamanho de fonte
% \csletcs{listfontsize}{footnotesize}

%% Lista de Ilustrações (somente se houver muitas ilustrações)
\listoffigures%

%% Lista de Tabelas (somente se houver muitas tabelas)
\listoftables%

%% Lista de Abreviaturas e Siglas (inserir itens em ordem alfabética)
\begin{AcronymsList}%[Título Alternativo]%% Substitui o título padrão
\item[ABNT] Associação Brasileira de Normas Técnicas
\item[CTAN] \ENLang{Comprehensive \TeX\ Archive Network}
\item[NBR] Norma Brasileira
\item[PDF] Formato de Documento Portátil (\ENLang*{Portable Document Format})
\item[QR] Resposta Rápida (\ENLang*{Quick Response})
\item[TUG] \ENLang{\TeX\ Users Group}
\item[URL] Localizador Uniforme de Recursos (\ENLang*{Uniform Resource Locator})
\item[UTFPR] Universidade Tecnológica Federal do Paraná
\end{AcronymsList}

%% Lista de Símbolos (inserir itens em ordem de ocorrência)
\begin{SymbolsList}%[Título Alternativo]%% Substitui o título padrão
\item[u] Velocidade, \Unit{m/s}
\item[\beta] Amplitude de velocidade, \Unit{m/s}
\item[\pi] Constante circular (Pi), \Unit{rad}
\item[A] Área, \Unit{m^2}
\item[\mathrm{Re}] Número de Reynolds
\end{SymbolsList}

%% Sumário (somente se houver muitas seções)
\tableofcontents%


\section{Introdução}%
\label{sect:intro}
A era digital e a rápida evolução tecnológica sublinham a importância crítica da educação em Ciência, Tecnologia, Engenharia, Artes e Matemática (STEAM). Para preparar as futuras gerações para um mercado de trabalho em constante mudança, onde uma parcela significativa dos empregos ainda não existe, sendo necessário fomentar o interesse e as habilidades em STEAM desde cedo. Nesse contexto, a robótica educacional surge como um campo de estudo e aplicação com potencial transformador \cite{meegleYouthRobotics2025}.

\subsection{A Crescente Importância da Educação STEAM e da Robótica para Jovens}
Ambientes de aprendizagem STEAM informais, como feiras de tecnologia e mostras científicas, desempenham um papel vital ao complementar a educação formal, impulsionando o interesse e o engajamento em campos STEAM \cite{ross2024BeyondExhibits}. A robótica educacional (RE) é amplamente reconhecida como uma ferramenta pedagógica eficaz que direciona a atenção e a motivação dos estudantes para as áreas STEAM \cite{cheung2024SummerLibraries}. Pesquisas demonstram que as atividades de RE não apenas promovem o interesse em tópicos relacionados a STEAM \cite{ross2024BeyondExhibits}, mas também aprimoram as atitudes em relação à engenharia e tecnologia \cite{cheung2024SummerLibraries}. A aprendizagem integrada de STEAM, que combina duas ou mais disciplinas STEAM em uma experiência conjunta, é particularmente eficaz, ajudando os alunos a fazer conexões entre as áreas, aprimorar habilidades de resolução de problemas e melhorar a alfabetização STEAM e a prontidão para a força de trabalho \cite{ross2024BeyondExhibits}.

O engajamento com a robótica vai além do conhecimento técnico, promovendo o desenvolvimento de habilidades essenciais para o século XXI, incluindo resolução de problemas, pensamento crítico, criatividade, colaboração, pensamento computacional e comunicação \cite{meegleYouthRobotics2025}. A experiência prática e interativa, baseada na teoria construtivista \cite{ross2024BeyondExhibits}, permite que os alunos apliquem conceitos teóricos a problemas práticos do mundo real \cite{meegleYouthRobotics2025}. A natureza \emph{divertida} e \emph{interessante} das atividades robóticas \cite{cheung2024SummerLibraries} serve como um motivador inicial, que se transforma em um engajamento mais profundo e sustentado, mesmo diante da dificuldade percebida. Esta abordagem ativa e prática é fundamental para o desenvolvimento cognitivo e social, pois incentiva a experimentação, a resolução de problemas e a colaboração.

Além do engajamento geral em STEAM, a robótica e a IA oferecem benefícios únicos para o desenvolvimento infantil, incluindo a aprendizagem socioemocional e habilidades de comunicação, especialmente para crianças neurodivergentes. Robôs sociais, como os modelos Pepper e NAO da SoftBank Robotics, são utilizados em ambientes educacionais e clínicos para ensinar regulação emocional, habilidades sociais e empatia. Esses robôs proporcionam um ambiente livre de julgamentos para as crianças praticarem interações sociais cruciais \cite{behavioralHealthRoboticsAI2025}. Estudos demonstram resultados promissores no apoio a crianças com autismo e outros transtornos de desenvolvimento (como será visto com mais detalhes na seção \nameref{sec:neurodivergentes}). % por exemplo, um estudo da Universidade de Yale observou melhorias notáveis no contato visual e na iniciação da comunicação entre crianças autistas que usaram um robô por 30 minutos diários. O robô Milo, projetado para crianças autistas, alcançou 87,5\% de engajamento, significativamente superior aos 2-3\% observados com terapeutas humanos, ilustrando a eficácia da tecnologia como um complemento aos cuidados humanos \cite{behavioralHealthRoboticsAI2025}.

\subsection{O Papel dos Ambientes de Aprendizagem Informal (Feiras de Tecnologia e Mostras Científicas)}

Espaços físicos projetados, como museus e centros de ciência, facilitam a aprendizagem exploratória e aberta, diferenciando-se dos ambientes de sala de aula tradicionais \cite{ross2024BeyondExhibits}. A educação STEAM informal é caracterizada por sua natureza voluntária, breve e emergente, permitindo que os participantes escolham suas experiências de aprendizagem e observem diretamente a relevância e aplicação do conhecimento STEAM em situações da vida real \cite{cheung2024SummerLibraries}. Esses ambientes são cruciais para despertar uma paixão duradoura pelo aprendizado e para fomentar a curiosidade, o pensamento crítico e a criatividade por meio de atividades práticas e envolventes \cite{meegleYouthRobotics2025}.
A capacidade desses ambientes informais de capturar o interesse fora das estruturas convencionais os posiciona como portas de entrada eficazes para o STEAM. Eles não são meramente atividades suplementares, mas componentes essenciais do pipeline de educação STEAM. A sua singularidade em despertar a curiosidade e o engajamento, de forma mais flexível e menos estruturada do que o ensino formal, torna-os ideais para alcançar públicos diversos e servir como um ponto de partida crítico para o envolvimento com STEAM. Isso implica a necessidade de parcerias estratégicas entre provedores de aprendizagem informal, instituições de ensino formal e a indústria, a fim de criar caminhos contínuos desde a inspiração inicial até a busca sustentada de estudos e carreiras em STEAM.


%\subsection{Objetivos e Estrutura do artigo}
%Este trabalho tem como objetivo analisar a aplicação de braços robóticos em feiras de tecnologia e mostras científicas como uma ferramenta para motivar, incentivar e despertar o interesse de crianças e adolescentes na pesquisa robótica. Para tanto, o documento apresentará uma revisão bibliográfica detalhada sobre o impacto da robótica educacional, orientações para a seleção e prototipagem de cenários, metodologias para avaliação do engajamento e análise de dados, documentação de melhores práticas e a proposição de diretrizes para uma implementação eficaz e segura.


\section{Revisão Bibliográfica: Impacto da Robótica Educacional no Engajamento STEAM de Jovens}
A literatura científica demonstra o papel transformador da robótica educacional (RE) no engajamento de jovens com as disciplinas STEAM. Esta importância será apresentada nos próximos tópicos.

\subsection{Benefícios da Robótica Prática para o Interesse e Motivação em STEAM}
As atividades de robótica educacional são reconhecidas como ferramentas eficazes para direcionar a atenção e a motivação dos estudantes para os campos STEAM \cite{cheung2024SummerLibraries}. Pesquisas indicam que a RE pode fomentar significativamente o interesse em tópicos relacionados a STEAM \cite{ross2024BeyondExhibits}. Por exemplo, estudos mostram que atividades práticas de design robótico, especialmente com plataformas como Arduino, despertam o interesse dos alunos em relação à engenharia e tecnologia \cite{cheung2024SummerLibraries}.

Os estudantes que participam de atividades robóticas frequentemente as descrevem como \emph{divertidas, interessantes, diferentes, difíceis, complexas e demoradas} \cite{cheung2024SummerLibraries}. Apesar da dificuldade e complexidade percebidas, uma grande proporção de estudantes apoia o uso da robótica, indicando que a natureza envolvente das atividades supera os desafios, impactando positivamente sua motivação e participação ativa \cite{cheung2024SummerLibraries}. Essa percepção inicial de diversão e interesse atua como um poderoso motivador, estimulando a curiosidade para o engajamento, mesmo quando as tarefas se tornam mais desafiadoras. O prazer derivado da interação com a robótica serve como um amortecedor psicológico, incentivando a persistência e a participação ativa, transformando a apreensão inicial em um envolvimento mais profundo.

A natureza prática da robótica resulta em resultados de aprendizagem superiores. Estudos sugerem que os alunos retêm significativamente mais informações (até 70\% mais) ao aprender por meio de atividades robóticas práticas, em comparação com métodos tradicionais baseados em palestras \cite{acebottImportanceRobotics2025}. Isso está alinhado com a teoria de aprendizagem construtivista, que enfatiza práticas ativas e interativas \cite{ross2024BeyondExhibits}. Além disso, os robôs educacionais são eficazes na promoção da aprendizagem integrada de STEAM, combinando conceitos de duas ou mais disciplinas STEAM em uma experiência conjunta. Essa abordagem interdisciplinar ajuda os alunos a fazer conexões entre as disciplinas, aprimora as habilidades de resolução de problemas e melhora a alfabetização STEAM e a prontidão para a força de trabalho \cite{ross2024BeyondExhibits}.

\subsection{Desenvolvimento de Habilidades do Século XXI -- Resolução de Problemas, Pensamento Crítico, Criatividade, Colaboração}

Os programas de robótica são fundamentais para o cultivo do pensamento crítico, da criatividade e da colaboração por meio de experiências de aprendizagem imersivas e práticas \cite{meegleYouthRobotics2025}. Evidências empíricas apoiam essas afirmações: estudantes envolvidos em projetos de robótica demonstraram uma melhoria de 25\% em suas habilidades de resolução de problemas em comparação com seus pares \cite{acebottImportanceRobotics2025}. Eles também exibem maior inovação, gerando em média de 3 a 5 ideias criativas por projeto \cite{acebottImportanceRobotics2025}. Além disso, a participação em equipes de robótica leva à melhoria das habilidades de comunicação e trabalho em equipe \cite{acebottImportanceRobotics2025}. Programas mantidos pela \textit{For Inspiration and Recognition of Science and Technology} (FIRST), como \textit{First Lego League} (FLL), \textit{FIRST Tech Challenge} (FTC) e \textit{FIRST Robotics Competition} (FRC) se apoiam nesses pilares \cite{firstRobotics2025}.

Um resultado significativo das atividades de RE é o aprimoramento das habilidades de pensamento computacional, que são reconhecidas tanto como um pré-requisito quanto como um benefício direto da educação STEAM habilitada por robótica \cite{rahman2024DigitalK12HRI} A robótica, por sua própria natureza, integra perfeitamente as disciplinas STEAM, oferecendo experiências de aprendizagem \emph{hands-on} e \emph{minds-on} \cite{cheung2024SummerLibraries}. Por exemplo, o design de um braço robótico envolve princípios de engenharia mecânica \cite{acebottImportanceRobotics2025}, a programação requer lógica de ciência da computação \cite{whalesbotRevolutionizingSTEM2025}, e a análise de movimento pode envolver matemática e física \cite{acebottImportanceRobotics2025}. Essa interconexão inerente aborda uma lacuna reconhecida na educação STEAM tradicional, que muitas vezes carece de integração interdisciplinar \cite{lim2024IntegratedSTEMRobotics}. Ao conceber demonstrações de braços robóticos, é fundamental destacar e conectar explicitamente esses aspectos interdisciplinares. Os cenários devem ilustrar como ciência, tecnologia, engenharia e matemática não são disciplinas isoladas, mas convergem em aplicações robóticas práticas, preparando os alunos para os desafios complexos e interconectados prevalentes nas carreiras STEAM modernas.

\subsection{Eficácia dos Ambientes de Aprendizagem STEAM Informais}
Museus e centros de ciência, como o \textit{Museum of Science} \cite{mosTeachingSTEM2025} e o \textit{Fleet Science Center} \cite{fleetScienceCenter2025}, são projetados para oferecer exposições e atividades ricas em STEAM que cultivam o amor pelo aprendizado para além das paredes da sala de aula convencional \cite{mosTeachingSTEM2025}. Esses ambientes informais proporcionam oportunidades de aprendizagem abertas, exploratórias e voluntárias, permitindo que os alunos vejam a relevância direta e a aplicação do conhecimento STEAM em contextos da vida real. Isso, por sua vez, aumenta seu interesse em STEAM e os incentiva a seguir carreiras relacionadas \cite{ross2024BeyondExhibits}. A \textit{National Science Foundation} (NSF) reconhece a educação científica informal (ISE) por seu papel vital no aumento do interesse, engajamento e compreensão da ciência, tecnologia, engenharia e matemática entre indivíduos de todas as idades \cite{informalscienceFramework2008}.

\subsection{Abordagem das Diversas Necessidades de Aprendizagem, Incluindo Crianças Neurodivergentes}
\label{sec:neurodivergentes}
A robótica e a inteligência artificial possuem um potencial substancial para transformar o desenvolvimento infantil, nutrindo habilidades essenciais como resolução de problemas, comunicação eficaz e regulação emocional \cite{behavioralHealthRoboticsAI2025}. Robôs sociais, incluindo modelos como \emph{Pepper} e \emph{NAO} da \textit{SoftBank Robotics}, são ativamente utilizados em ambientes educacionais e clínicos para ensinar regulação emocional, habilidades sociais e empatia. Esses robôs fornecem um ambiente único, livre de julgamentos, para as crianças praticarem essas interações sociais cruciais \cite{behavioralHealthRoboticsAI2025}.

Estudos mostram um potencial inspirador no apoio a crianças com autismo e outros transtornos de desenvolvimento. Por exemplo, um estudo da Universidade de Yale observou melhorias notáveis em habilidades como contato visual e iniciação da comunicação entre crianças autistas que usaram um robô por 30 minutos por dia. O robô \emph{Milo}, projetado para crianças autistas, alcançou 87,5\% de engajamento, significativamente maior do que o engajamento com terapeutas humanos (2-3\%), demonstrando a eficácia da tecnologia como um suplemento aos cuidados humanos \cite{behavioralHealthRoboticsAI2025}. Este impacto se estende além do aprendizado técnico de STEAM. Ao incorporar elementos que promovem o desenvolvimento de habilidades socioemocionais e de comunicação, como desafios robóticos colaborativos ou demonstrações de robótica assistiva, os designers de exposições e educadores podem ampliar a proposta de valor da robótica. Isso a torna relevante para uma gama mais ampla de objetivos educacionais e de desenvolvimento, atraindo um público mais diversificado e maximizando o impacto geral do programa.

\section{Prototipagem de Cenários e Seleção de Braços Robóticos para Engajamento Público}
A seleção cuidadosa de braços robóticos e a prototipagem de cenários interativos são fundamentais para maximizar o impacto educacional em feiras e exposições. Esta seção aborda alguns alguns desses cuidados. Contudo, diferentes fabricantes, tem propostas educacionais divergentes da aplicação de seus produtos. Essas divergências dificultam a escolha de um braço robótico \emph{ideal} para ampla aplicação em educação STEAM, abrindo oportunidades de pesquisa para a modelagem de um braço robótico com material educacional que possa atender a educação STEAM e aproximar da realidade da indústria.

\subsection{Características de Braços Robóticos Adequados para Demonstrações Educacionais}
Para demonstrações educacionais, os braços robóticos devem possuir características específicas que garantam a eficácia e a segurança do engajamento de crianças e adolescentes:
	\begin{description}
		\item [Facilidade de Uso]: Os braços robóticos devem ser intuitivos e fáceis de operar para o público jovem. Isso inclui suporte a vários níveis de programação, desde codificação baseada em blocos para iniciantes até Python para usuários mais avançados \cite{makeblockMBot22025}. Exemplos de plataformas amigáveis incluem mBot2 \cite{makeblockMBot22025}, LEGO Mindstorms EV3 \cite{centrePointRoboticsGames2025}, VEX Robotics \cite{meegleYouthRobotics2025} e vários kits baseados em Arduino \cite{cheung2024SummerLibraries}. O Dobot Magician, por exemplo, oferece diversos métodos de controle, como EEG, Bluetooth, WiFi, Mobile, PC, Gesto e Joystick, juntamente com interfaces de programação gráfica \cite{robotlabDobotClassroomPack2025}. A capacidade de oferecer diferentes modos de interação garante que o braço robótico seja acessível a uma ampla faixa etária, desde crianças pequenas que podem operar com um joystick até adolescentes que podem programar o mesmo braço com código.
		\item [Apelo Visual]: O design do braço robótico pode influenciar significativamente o engajamento. Embora opções DIY simples, utilizando materiais como papelão ou palitos de picolé, possam ser eficazes para a construção prática \cite{scienceBuddiesBuildRoboticArm2025}, designs mais sofisticados e visualmente atraentes, incluindo robôs biomiméticos (inspirados em movimentos de animais) \cite{ross2024BeyondExhibits}, podem cativar imediatamente e criar uma conexão com o público jovem \cite{chang2025ConstructedResponse}. Um braço que se assemelha a um membro humano ou animal, por exemplo, pode gerar uma curiosidade natural e um desejo de interagir.
		\item [Custo-Benefício]: A acessibilidade é um desafio chave na robótica educacional \cite{ross2024BeyondExhibits}. Há uma clara necessidade de robôs educacionais de baixo custo e fáceis de usar \cite{ross2024BeyondExhibits}. Kits acessíveis como Adeept (com preços em torno de US\$ 70-80) \cite{robotshopArmsGrippers2025} e opções DIY (por exemplo, projetos baseados em Arduino usando materiais comuns) \cite{scienceBuddiesBuildRoboticArm2025} são viáveis. Para recursos mais avançados, braços educacionais como o Dobot Magician são considerados com bom custo-benefício, dadas suas capacidades \cite{robotlabDobotClassroomPack2025}. Robôs colaborativos (cobots) de fabricantes como a Universal Robots também estão se tornando mais acessíveis economicamente para fins educacionais e de exibição pública \cite{top3dshopDobotMagicianReview2023}.
		\item [Durabilidade e Segurança]: Para interação pública contínua, a robustez e a segurança são primordiais \cite{ross2024BeyondExhibits}. Robôs colaborativos (cobots) são projetados especificamente para interação segura humano-robô, incorporando recursos como limitação de potência e força, monitoramento de velocidade e separação, materiais de construção leves e bordas arredondadas \cite{top3dshopDobotMagicianReview2023}. A adesão a padrões internacionais de segurança (por exemplo, ISO 10218, ISO 13482, ANSI/RIA R15.06, UL 1740) é necessária para qualquer robô operando em espaços públicos \cite{standardBotsCobotSafetyStandards2025}. Embora a qualidade de construção do Dobot Magician seja geralmente boa para uso industrial leve, a durabilidade específica a longo prazo para interação pública contínua não é extensivamente detalhada \cite{robotlabDobotClassroomPack2025}. A durabilidade dos cabos também é uma consideração de design para alguns braços robóticos \cite{bostonDynamicsSpot2025}.
		\item [Modularidade e Expansibilidade]: Kits robóticos que permitem configurações versáteis e suportam a integração de sensores e módulos adicionais (por exemplo, Makeblock mBot Ultimate, VEX Robotics, Dobot Magician) \cite{makeblockMBot22025} permitem uma gama mais ampla de atividades interativas e adaptabilidade a diferentes objetivos de aprendizagem, garantindo que o investimento no equipamento possa ser aproveitado em diversas demonstrações.
	\end{description}


A concepção de exposições com braços robóticos deve considerar a criação de níveis de interação e complexidade variados. Uma demonstração básica pode permitir o controle direto por joystick para crianças mais novas, enquanto uma estação adjacente pode permitir que estudantes mais velhos programem o mesmo braço usando código baseado em blocos ou Python. Essa abordagem em camadas garante que a exposição permaneça envolvente e educacional em uma ampla faixa etária, facilitando uma jornada de aprendizagem progressiva e maximizando o impacto geral.

\subsection{Atividades Interativas e Desafios para Braços Robóticos}
A escolha de atividades interativas é crucial para o engajamento e a aprendizagem. As seguintes categorias de atividades podem ser implementadas:
	\begin{itemize}
		\item Tarefas de \emph{Pick-and-Place}: Essas são atividades fundamentais de robótica e altamente envolventes para crianças. Podem envolver a classificação de objetos por cor ou forma, ou a organização de itens em uma sequência, simulando processos de linha de montagem \cite{robotlabDobotClassroomPack2025}.
		\item Desenho/Escrita: Braços robóticos equipados com suportes de caneta podem ser programados para desenhar textos personalizados, padrões intrincados ou até mesmo replicar obras de arte famosas, fomentando a criatividade e demonstrando controle de precisão \cite{centrePointRoboticsGames2025}.
		\item Processos Industriais Simulados: As exposições podem imitar a automação de fábricas do mundo real, utilizando braços robóticos com esteiras transportadoras e vários sensores (por exemplo, fotoelétricos, sensores de cor) para classificar, montar ou embalar itens. Isso proporciona uma compreensão tangível das aplicações industriais e da eficiência da robótica \cite{robotlabDobotClassroomPack2025}.
		\item Resolução de Labirintos/Planejamento de Trajetória: Os alunos podem programar braços robóticos para navegar em um labirinto definido ou pegar objetos ao longo de um caminho complexo, evitando obstáculos \cite{rancholabsUltimateGuide2025}. Isso introduz conceitos como cinemática, cinemática inversa e algoritmos básicos, mostrando como os robôs \emph{pensam} para realizar tarefas \cite{chang2025ConstructedResponse}.
		\item Desafios de Biomimética: Projetar braços robóticos que emulam movimentos ou funções observadas em animais (por exemplo, a mecânica da mandíbula de uma cobra, o lançamento de teias de uma aranha ou a escalada de um camaleão) pode ser altamente criativo e educacional, conectando a biologia à engenharia \cite{ross2024BeyondExhibits}. 
		\item Construção Simples de Mãos Robóticas: Atividades envolventes podem incluir a construção de uma mão robótica básica usando materiais domésticos facilmente disponíveis, como canudos e barbante, para demonstrar princípios fundamentais de preensão e manipulação \cite{scienceBuddiesRoboticsProjects2025}.
		\item Jogos Colaborativos: Projetar jogos interativos onde humanos e robôs cooperam ou competem pode aumentar o engajamento. Exemplos incluem \emph{Corrida de Robôs} (competir contra um braço robótico em um jogo) ou jogos de \emph{Cooperação}, que podem ser vagamente ligados a aplicações do mundo real, como robôs cirúrgicos \cite{centrePointRoboticsGames2025}.
	\end{itemize}



É fundamental que os cenários com braços robóticos sejam cuidadosamente elaborados em torno de problemas ou aplicações do mundo real que sejam relevantes para os jovens. Em vez de demonstrações abstratas de movimento, as exposições devem apresentar desafios que imitem tarefas industriais, tecnologias assistivas ou até mesmo empreendimentos artísticos criativos. Essa contextualização aumenta a motivação intrínseca, aprofunda a compreensão e conecta diretamente a robótica a possíveis caminhos de carreira e contribuições sociais mais amplas, tornando a experiência de aprendizagem mais impactante e memorável.

\subsection{Integração de Conceitos de IA e Aprendizado de Máquina em Demonstrações}
A inclusão de conceitos de Inteligência Artificial (IA) e Aprendizado de Máquina (ML) em demonstrações de braços robóticos eleva o nível de engajamento e a profundidade da aprendizagem. Robôs com capacidades de IA integradas oferecem experiências interativas avançadas, como reconhecimento de fala, controle por gestos e reconhecimento de objetos \cite{vexRoboticsHome2025}.

As demonstrações podem apresentar robôs de aprendizagem conversacionais alimentados por IA (por exemplo, Miko) \cite{mikoAIPoweredRobot2025}, carros autônomos em miniatura \cite{stempediaProjects2025} ou sistemas que podem reconhecer e categorizar objetos, como brinquedos \cite{stempediaProjects2025}. Os alunos podem ser introduzidos aos conceitos fundamentais do aprendizado de máquina treinando um robô para reconhecer padrões (por exemplo, gestos de mão) e aplicando esse aprendizado para controlar as ações do robô \cite{stempediaProjects2025}. Plataformas de aprendizagem virtual também podem fornecer um ambiente seguro e acessível para explorar conceitos de robótica com IA sem a necessidade de hardware físico \cite{coderoboPickPlace2025}.

\subsection{Protocolos de Segurança para Interação Humano-Robô em Espaços Públicos}
A segurança é a preocupação primordial em qualquer demonstração pública envolvendo robótica, especialmente com crianças:

	\begin{description}
		\item [Priorizar Robôs Colaborativos (Cobots)]: Para demonstrações públicas que envolvem interação direta com crianças, os robôs colaborativos (cobots) são a escolha preferencial. Esses robôs são projetados especificamente para interação segura humano-robô em espaços de trabalho compartilhados, contando com recursos de segurança intrínsecos, como construção leve, bordas arredondadas e limitações de velocidade e força \cite{top3dshopDobotMagicianReview2023}.
		
		\item [Implementar Recursos Essenciais de Segurança]:
			\begin{itemize}
				\item Limitação de Potência e Força: Os cobots devem ser configurados para limitar sua saída de força, garantindo que, mesmo em caso de contato acidental, o risco de lesão seja minimizado \cite{wikipediaCobot2025}.
				\item Monitoramento de Velocidade e Separação: A velocidade do robô deve se ajustar dinamicamente com base na proximidade de humanos, diminuindo ou parando se uma pessoa entrar em uma zona de segurança predefinida \cite{wikipediaCobot2025}.
				\item Paradas de Emergência Facilmente Acessíveis: Botões de parada de emergência claramente marcados e de fácil acesso (por exemplo, botões de palma, cordas de puxar) devem ser estrategicamente posicionados em todas as zonas de interação, anulando todos os outros controles \cite{kinovaJacoAssistiveTechnologies2025}.
				\item Zonas Seguras Programáveis: Definir e programar áreas específicas onde o movimento do robô é restrito, especialmente em ambientes públicos lotados \cite{teradyneCollaborativeRobots2025}.
				\item Barreiras Físicas/Proteções (conforme necessário): Embora os cobots visem a interação direta, uma avaliação de risco completa ainda pode exigir o uso de barreiras intertravadas ou proteções fixas para certos movimentos de alto risco ou para braços robóticos mais pesados/rápidos. Barreiras de conscientização (por exemplo, grades baixas) podem definir perímetros em cenários com riscos mínimos \cite{kinovaJacoAssistiveTechnologies2025}.
			\end{itemize}
		\item Adesão a Padrões e Certificações de Segurança: A conformidade com os padrões internacionais de segurança (por exemplo, ISO 10218, ISO 13482, ANSI/RIA R15.06, UL 1740) é inegociável para robôs que operam em espaços públicos \cite{standardBotsCobotSafetyStandards2025} Organizações e empresas são legal e eticamente responsáveis pela operação segura de seus siSTEAMas robóticos \cite{li2023LowCostCableDrivenArm}.
		\item Treinamento Abrangente de Operadores: Todo o pessoal envolvido na operação ou supervisão de demonstrações de braços robóticos deve receber treinamento completo em protocolos de segurança, procedimentos operacionais e planos de resposta a emergências \cite{kinovaJacoAssistiveTechnologies2025}.
		\item Avaliação Rigorosa de Riscos: Uma avaliação de risco detalhada e específica para cada exposição deve ser realizada para identificar todos os perigos potenciais e implementar salvaguardas apropriadas e em várias camadas \cite{top3dshopDobotMagicianReview2023}.
		\item Durabilidade e Manutenção: Selecionar braços robóticos conhecidos por sua robustez e durabilidade para uso público contínuo é fundamental. Embora dados específicos de durabilidade a longo prazo para todos os robôs educacionais possam ser limitados, a escolha de modelos com boa qualidade de construção (por exemplo, Dobot Magician para uso industrial leve) \cite{robotlabDobotClassroomPack2025} e a garantia da durabilidade de componentes a longo prazo (por exemplo, cabos) \cite{bostonDynamicsSpot2025} são questões para determinar a melhor escolha do equipamento em campo.
	\end{description}
	

\section{Considerações}
A aplicação estratégica de braços robóticos em feiras de tecnologia e mostras científicas oferece um potencial imenso para motivar e engajar crianças e adolescentes na pesquisa robótica e nos campos STEAM em geral. A análise apresentada neste relatório sublinha que o sucesso dessas iniciativas depende de uma abordagem multifacetada que integra o design de exposições envolventes, a seleção de tecnologias apropriadas, a implementação de protocolos de segurança rigorosos e a avaliação contínua do impacto.

A natureza prática e interativa da robótica educacional não apenas desperta o interesse inicial, mas também fomenta o desenvolvimento de habilidades cruciais para o século XXI, como pensamento crítico, resolução de problemas e colaboração. A capacidade de contextualizar a aprendizagem em cenários do mundo real e de adaptar as atividades a diversas faixas etárias e necessidades de aprendizagem amplifica o impacto educacional. Além disso, a robótica oferece uma plataforma única para abordar o desenvolvimento socioemocional e promover a inclusão, especialmente para crianças neurodivergentes.

\section{Proposta de pesquisa}
Para otimizar ainda mais a aplicação de braços robóticos em ambientes educacionais informais, as seguintes recomendações são propostas:

\printbibliography%% Referências


\end{document}
